% !TEX root =  ../Dissertation.tex

\chapter{Methodology}

This section depicts the methodology used for the present work by walking through a motivating example, and defining the method used afterwards.

% Example????

\section{A Motivating Example: Grid World}

The Grid World is a simple toy example (see Figure \ref{}), with the following properties in terms of a MDP (See Section \ref{sec:mdp_definition}):

\begin{itemize}
    \item The state space is discrete an given by $ s \in S : \{0,n\} \times \{0,m\}$, all the positions $(x,y)$ of the agent in the grid.
    \item The actions space is discrete given by $a \in A: \{0,3\}$, which corresponds to four possible directions: down (0), right (1), up (2) and left (4).
    \item The transition are deterministic and only restricted to adjacent cells, such that the agent will move to any adjacent cell with probability 1 $P(s,a) = 1$ (except if this is a wall when it keeps in the same state, or transition to the same state with probability 1).
    \item The rewards are binary and sparse, with the immediate reward always being -1 unless the agent reaches the goal (G), in which case it becomes 200.
    \item And the discount factor $\gamma$ is 0.99
\end{itemize}

According to Definiton \ref{}, a bisimulation must hold two constraints the Reward Equivalence and Transition Probability Equivalence. 



In the Grid World, the Reward Equivalence holds by

$$$$


\subsection{Bisimulation in a Grid World}

\subsection{On-policy Bisimulation in a Grid World}

\subsection{On-policy Bisimulation metric in a Grid World}

%\subsection{On-policy MICO metric justification}

\section{Bisimulation Prioritized Experience Replay}

\subsection{Diffuse metric}
\subsection{Learning State Abstractions}
\subsection{Priority Strategies}


\section{Experimentation Setup}

\subsection{Experiments and Environments}
\subsection{TorchRL}
