% !TEX root =  ../Dissertation.tex

\chapter{Background}


\section{Equivalence Relation}


\begin{definition}[Equivalence Relation]
Let $X$ be a set. An equivalence relation on $X$ is a subset $R \subseteq X \times X$ that satisfies the following three properties:
\begin{enumerate}
    \item \textbf{Reflexivity}: For all $x \in X$, $(x,x) \in R$;
    \item \textbf{Symmetry}: For all $x, y \in X$, if $(x,y) \in R$ then $(y,x) \in R$;
    \item \textbf{Transitivity}: For all $x, y, z \in X$, if $(x,y) \in R$ and $(y,z) \in R$ then $(x,z) \in R$.
\end{enumerate}
\end{definition}

In mathematics, an equivalence relation is a binary relation that is reflexive, symmetric and transitive. A simpler example is equality. 

\begin{enumerate}
    \item Any number $a$ is equal to itself (reflexive).
    \item  If $a=b$, then $b=a$ (symmetric).
    \item If $a=b$ and $b=c$, then $a=c$ (transitive). 
\end{enumerate}

\subsection{}