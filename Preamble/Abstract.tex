
\chapter*{Abstract}
\addcontentsline{toc}{chapter}{Abstract}

% Prioritized Experience Replay has been an effective traditional solution to valued-based algorithms to efficiently address non-stationary and correlated data issues. However, standard prioritization often overlooks the nuanced, task-specific behaviors of states, leading to a "task-agnostic" sampling problem. This work introduces a novel approach by incorporating a surrogate on-policy bisimulation metric into the experience replay prioritization process. This metric allow us to measure behavioral similarities and diversifies the training data, aiming to enhance learning by focusing on behaviorally relevant transitions. Specifically, our method utilizes a Matching under Independent Couplings (MICo) metric, a more general surrogate metric learned through state abstractions. The proposed method balances the conventional TD-error-based and bisimulation-based prioritization, by reweighting the priorities with the introduction of an hyperparameter, enriching the training process with more informative state transitions. The method demonstrates outperforming performance in a 31-sates Grid World used to empirically validate the method, efficiently achieving to 1) emphasize behavioral relevant transition, thereby avoiding task-agnostic sampling, 2) alleviate the outdated priorities by having a better tendency to constant fixed priorities, and 3) mitigate the insufficient sample space coverage, increasing the data diversity. Additionally, it show promising results when evaluated in MountainCar, CartPole, Acrobot and Lunar Lander, with certain consideration to successfully implemented them in practice.


Prioritized Experience Replay has been an effective traditional solution for value-based reinforcement learning algorithms to efficiently address non-stationary and correlated data issues. However, standard prioritization often overlooks the nuanced, task-specific behaviors of states, leading to a "task-agnostic" sampling problem. This work introduces a novel approach by incorporating a surrogate on-policy bisimulation metric into the experience replay prioritization process. This metric allows us to measure behavioral similarities and diversify the training data, aiming to enhance learning by focusing on behaviorally relevant transitions. Specifically, our method utilizes a Matching under Independent Couplings (MICo) metric, a more general surrogate metric learned through state abstractions. The proposed method balances conventional TD-error-based and bisimulation-based prioritization by reweighting priorities with an introduced hyperparameter, and two possible strategies, BPERcn and BPERaa. The method demonstrates superior performance in a 31-state Grid World, which was used for empirical validation, efficiently achieving: 1) emphasis on behaviorally relevant transitions, thereby avoiding task-agnostic sampling; 2) alleviation of outdated priorities by maintaining more consistent fixed priorities; and 3) mitigation of insufficient sample space coverage by increasing data diversity. Additionally, it shows promising results when evaluated in pixel-based environments, such as MountainCar, CartPole, Acrobot, and Lunar Lander, with some additional considerations for successful practical implementation.

% , it enriches the training process with more informative state transitions