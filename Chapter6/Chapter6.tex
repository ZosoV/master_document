% !TEX root =  ../Dissertation.tex

\chapter{Conclusion and Future Work}

Even though it returns promising results, one thing that I realized is that the bisimulation values are relative too high compare with the td-errors, which can produce to focus too much in some experiences and unven and drastically skewness probability distribution, a way to soft the sampling distribution could be recommended. 

We have tried to normalize using a logaritmic scale and max min, but it didn't worked properly.

Even when using the 0.01 beta in the loss function

Theoretical proof

It lacks of a theoretical basis

also some experimetns suggest that most of the improvements comes from the MICO and not actually the prioritization

a sweep over the alpha and beta values, although we use the best one for PER, they could not be the best ones for BPER

Additionally, a way to normalize the priorities or another sampling method like the rank based because the BPER priorities are larger than the td-error (ojo).

Try Double DQN and other algorithms and better benchmarks.

Check with more state of the art algorithms and methods


We hypothesis that earlier improvement on BPERcn and BPERaa are more related with randomly picking different distant states in the beginning. However, over time as the metric is learn, the BPERaa because a better indicator of bisimilar transtion (or more informative transition), particularly for the scarce and highly variable approximatation mention in [], and the motive for what Strategy 2 was actually proposed.

% VISA

% For visits of up to 6 months for most purposes.
% Cost: CAN \$100

% https://neurips.cc/Conferences/2024/Dates

% https://www.canada.ca/en/immigration-refugees-citizenship/services/visit-canada/about-visitor-visa.html

% https://www.canada.ca/content/dam/ircc/documents/pdf/english/kits/forms/imm5645/01-01-2021/imm5645e.pdf

% https://www.canada.ca/en/immigration-refugees-citizenship/services/visit-canada/apply-visitor-visa.html